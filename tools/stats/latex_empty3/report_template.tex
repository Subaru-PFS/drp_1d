%% LyX 2.2.0 created this file.  For more info, see http://www.lyx.org/.
%% Do not edit unless you really know what you are doing.
\documentclass[12pt,english]{scrreprt}
\usepackage[T1]{fontenc}
\usepackage[utf8]{inputenc}
\setcounter{secnumdepth}{3}
\setcounter{tocdepth}{3}
\usepackage{float}
\usepackage{amsmath}
\usepackage{amssymb}
\usepackage{graphicx}
\usepackage{setspace}
\doublespacing

\makeatletter
%%%%%%%%%%%%%%%%%%%%%%%%%%%%%% User specified LaTeX commands.
\usepackage[T1]{fontenc}
\usepackage{mathdots}
\usepackage{lipsum}% juste pour faire du remplissage
\usepackage{fancyhdr}
\usepackage{url}
\usepackage{caption}
\usepackage{algorithm}
\usepackage[toc,page]{appendix} 
\usepackage{moreverb}
\usepackage{setspace}
\usepackage{longtable}

%\usepackage[utf8]{inputenc}
\usepackage{minted}

%VARIABLES
\def\title{Amazed output statistics/performance report}
\def\texttitleb{unused}
\def\texttitlec{-}
\def\textversion{Version: x.x}
\def\textdate{Date: tpl_tobereplaced_formatteddate}
\def\textproject{Project: AMAZED}
\def\author{Automatically generated by processAmazedOutputStats.py}

\renewcommand{\appendixtocname}{Annexes} 
\renewcommand{\appendixpagename}{Annexes} 
%\usepackage{titlesec}
%\titleformat{\chapter}[display]
%{\normalfont\huge\bfseries}{\chaptertitlename\ \thechapter}{20pt}{\Huge}
\DeclareCaptionFormat{myformat}{#3}
\captionsetup[algorithm]{format=myformat}

% Commande qui redéfinie la façon dont un chapitre sera affiché
% par la commande \leftmark
\renewcommand{\chaptermark}[1]{%
   \markboth{%
      \slshape% pour écrire en lettres penchées (par exemple)
      Chapitre \thechapter\ : #1%
   }{}%
}

% Commande qui redéfinie la façon dont une section sera affichée
% par la commande \rightmark
\renewcommand{\sectionmark}[1]{%
   \markright{%
      \slshape% pour écrire en lettres penchées (par exemple)
      Section\ \thesection\ -- #1%
   }%
}

%try to redefine the chapter title formatting (less space before)
\usepackage{titlesec}
\titleformat{\chapter}[display]   
{\normalfont\huge\bfseries} % format
{} % label
{1pt} % sep
{\Huge \thechapter. }   
\titlespacing*{\chapter}{0pt}{-80pt}{5pt}
%%%%

%\renewcommand{\chaptername}{}
%\renewcommand{\thechapter}{}

% Pour les en-têtes, L = LEFT, C = CENTER et R = RIGHT.
% Enfin E = EVEN (pages paires) et O = ODD (pages impaires)
% Par exemple, l'option RE concerne l'entête des pages 
% paires à droite.
   
%\fancyhead[L,C,R]{} % On efface tout pour repartir de zéro.

% Pour les pages paires
%\fancyhead[LE]{\bfseries\thepage}
%\fancyhead[RE]{\leftmark}

% Pour les pages impaires
%\fancyhead[RO]{\bfseries\thepage}
%\fancyhead[LO]{\rightmark}



\usepackage{xcolor}\usepackage{array}\usepackage{float}\usepackage{units}\usepackage{setspace}


%%%%%%%%%%%%%%%%%%% HEADER %%%%%%%%%%%%%%%%%%%%%%%%%%
\usepackage[lmargin=0.5in,rmargin=0.5in,top=0.1in,headheight=2.5\baselineskip,headsep=1\baselineskip,includehead]{geometry}
%\addtolength{\textwidth}{0.5in}

\usepackage{tabularx}
\usepackage{graphicx}
%% some invisible "struts" to help define the structures and row heights.
\newcommand{\aevstrut}{\rule{0pt}{2.5ex}}
\newcommand{\aehstrut}{\rule{0.45em}{0pt}}


\newlength{\headerwidth}
\setlength{\headerwidth}{\textwidth}
\newsavebox{\myheader}
\begin{lrbox}{\myheader}%
    \begin{minipage}[b]{\headerwidth}
    \renewcommand{\arraystretch}{0.6}%
    \begin{tabularx}{\headerwidth}{|c|X|l|}\hline
                    &    & \small{\textdate}   \\\cline{3-3}
            \raisebox{\dimexpr0.73ex-0.5\totalheight\relax}[0pt][0pt]{\centering{\includegraphics[height=0.5in]{images/header/logo_cesam.png}}} 	& {\large{\bfseries{\title}}}   & {\small{\textversion}} \\\cline{3-3}
            & \texttitlec &\small{\textproject}     \\\cline{1-3}
            \multicolumn{3}{|l|}{\small{\author}} \\\hline
    \end{tabularx}
    \vspace{-0.5ex}\par
    \end{minipage}
\end{lrbox}

%% Setting up the header
\usepackage{fancyhdr}
%\pagestyle{fancy}
\fancypagestyle{plain}{
  \fancyhf{}% Clear header/footer
  \fancyhead[R]{\usebox{\myheader}}% Right header
  %\fancyfoot[L]{Name Firstname - v1.0 \\  Date}% Left footer
  \fancyfoot[C]{\thepage\ }% Right footer
}
\pagestyle{plain}% Set page style to plain
\renewcommand{\headrulewidth}{0pt}
\renewcommand{\footrulewidth}{0pt}
\usepackage{lipsum}
%%%%%%%%%%%%%%%%% END HEADER %%%%%%%%%%%%%%%%%%%%%%%%%%%%

%%%%%%%%%%%%%%%%%%%%%%%%%%%%%% LyX specific LaTeX commands.
\floatstyle{ruled}
\newfloat{algorithm}{tbp}{loa}
\providecommand{\algorithmname}{Algorithme}
\floatname{algorithm}{\protect\algorithmname}

%% Because html converters don't know tabularnewline
\providecommand{\tabularnewline}{\\}

%%%%%%%%%%%%     PAGE DE GARDE  PERSONALISÉE  %%%%%%%%%%%%%%%
\renewcommand{\maketitle}{
  \begin{titlepage}
  	\vspace*{80pt}
\vskip 20\p@
\begin{center}

%\LARGE \begin{bfseries} xx  \end{bfseries}  \par
%v0.1  \par
%xx \par
\LARGE {\bfseries \texttitlea}  \par
\texttitleb \par
\vskip 200\p@

\vskip 100\p@
\small \textversion  \par
\large \@author, \textdate\par
\end{center}
\vskip 35\p@

  \end{titlepage}
}
%%%%%%%%%%%%     /PAGE DE GARDE PERSONALISEE   %%%%%%%%%%%%%%%


\usepackage{babel}
\addto\extrasfrench{%
   \providecommand{\og}{\leavevmode\flqq~}%
   \providecommand{\fg}{\ifdim\lastskip>\z@\unskip\fi~\frqq}%
}

\makeatother

\usepackage{babel}
\begin{document}
%\tableofcontents{}

\chapter{Configuration}
\begin{center}
\vspace{20pt}
  \begin{tabular}{ l | p{10cm} }
    \hline
    Spectrum List & tpl_tobereplaced_param_spclist \\ \hline
    Spectrum Directory & tpl_tobereplaced_param_spcdir \\ \hline
    Spectra filter for stats & tpl_tobereplaced_param_binfilter \\ \hline
    Spectra count & tpl_tobereplaced_param_spccount \\ \hline
    Method & tpl_tobereplaced_param_method \\ \hline
    Continuum estimation & tpl_tobereplaced_param_continuummethod \\ \hline
    \hline
  \end{tabular}
\end{center}


\chapter{Performance matrices}

\begin{figure}[H]
\begin{centering}
\includegraphics[width=0.85\paperwidth]{tpl_tobereplaced_fig_performancesummary}
\par\end{centering}
\centering{}\caption{Performance summary. More details in the Annexe 1. \label{fig:perf_summary}}
\end{figure}


\chapter{Global Success Rate}

\begin{figure}[H]
\begin{centering}
\includegraphics[width=0.55\paperwidth]{tpl_tobereplaced_fig_successrateglobal}
\par\end{centering}
\centering{}\caption{Global Success Rate. For $\frac{|{z_{calc}-z_{ref}}|}{1+z_{ref}}$=tpl_tobereplaced_globalsuccessrateBinCenter, the success rate is tpl_tobereplaced_globalsuccessrateValue \%.\label{fig:successrate_global}}
\end{figure}

\chapter{Scatter plots}

\begin{figure}[H]
\begin{centering}
\includegraphics[width=0.50\paperwidth]{tpl_tobereplaced_fig_scatterplotglobal}
\includegraphics[width=0.50\paperwidth]{tpl_tobereplaced_fig_scatterplotzcalczrefglobal}
\par\end{centering}
\centering{}\caption{Global Scatter Plots. More details in the Annexe 2. \label{fig:scatter_global}}
\end{figure}

\chapter{Success Rates histograms}

\begin{figure}[H]
\begin{centering}
\includegraphics[width=0.55\paperwidth]{tpl_tobereplaced_fig_histversus_mag}
\par\end{centering}
\centering{}\caption{Success Rates histograms versus magnitude.\label{fig:successrate_vs_mag}}
\end{figure}

\begin{figure}[H]
\begin{centering}
\includegraphics[width=0.55\paperwidth]{tpl_tobereplaced_fig_histversus_redshift}
\par\end{centering}
\centering{}\caption{Success Rates histograms versus redshift.\label{fig:successrate_vs_redshift}}
\end{figure}

\begin{figure}[H]
\begin{centering}
\includegraphics[width=0.6\paperwidth]{tpl_tobereplaced_fig_histversus_sfr}
\par\end{centering}
\centering{}\caption{Success Rates histograms versus SFR.\label{fig:successrate_vs_sfr}}
\end{figure}

\begin{figure}[H]
\begin{centering}
\includegraphics[width=0.6\paperwidth]{tpl_tobereplaced_fig_histversus_logfhalpha}
\par\end{centering}
\centering{}\caption{Success Rates histograms versus $H_{alpha}$ integrated flux.\label{fig:successrate_vs_logfhalpha}}
\end{figure}

\begin{figure}[H]
\begin{centering}
\includegraphics[width=0.6\paperwidth]{tpl_tobereplaced_fig_histversus_ebmv}
\par\end{centering}
\centering{}\caption{Success Rates histograms versus E(B-V).\label{fig:successrate_vs_ebmv}}
\end{figure}

\begin{figure}[H]
\begin{centering}
\includegraphics[width=0.6\paperwidth]{tpl_tobereplaced_fig_histversus_sigma}
\par\end{centering}
\centering{}\caption{Success Rates histograms versus $Sigma_V$.\label{fig:successrate_vs_sigmav}}
\end{figure}



\chapter{Annexe 1}
...

\chapter{Annexe 2}

\begin{figure}[H]
\begin{centering}
\includegraphics[width=0.55\paperwidth]{tpl_tobereplaced_fig_scatterplotzoom0}
\includegraphics[width=0.55\paperwidth]{tpl_tobereplaced_fig_scatterplotzoom1}
\par\end{centering}
\centering{}\caption{Scatter Plots (zoom).\label{fig:scatter_zoom0_1}}
\end{figure}
\begin{figure}[H]
\begin{centering}
\includegraphics[width=0.55\paperwidth]{tpl_tobereplaced_fig_scatterplotzoom2}
\par\end{centering}
\centering{}\caption{Scatter Plots (zoom).\label{fig:scatter_zoom2}}
\end{figure}


\chapter{Annexe 3}
The methods parameters file used for this result-set is the following:
\inputminted[frame=lines,framesep=2mm,baselinestretch=1.2,fontsize=\small,linenos] 
{json}{tpl_tobereplaced_jsonparamsfullpath}


\end{document}
